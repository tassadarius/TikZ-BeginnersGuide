%%%%%%%%%%%%%%%%%%%%%%%%%%%%%%%%%%%%%%%%%
% Beamer Presentation
% LaTeX Template
% Version 1.0 (10/11/12)
%
% This template has been downloaded from:
% http://www.LaTeXTemplates.com
%
% License:
% CC BY-NC-SA 3.0 (http://creativecommons.org/licenses/by-nc-sa/3.0/)
%
%%%%%%%%%%%%%%%%%%%%%%%%%%%%%%%%%%%%%%%%%

%----------------------------------------------------------------------------------------
%	PACKAGES AND THEMES
%----------------------------------------------------------------------------------------

\documentclass{beamer}

\mode<presentation> {

% The Beamer class comes with a number of default slide themes
% which change the colors and layouts of slides. Below this is a list
% of all the themes, uncomment each in turn to see what they look like.

%\usetheme{default}
%\usetheme{AnnArbor}
%\usetheme{Antibes}
%\usetheme{Bergen}
%\usetheme{Berkeley}
%\usetheme{Berlin}
%\usetheme{Boadilla}
%\usetheme{CambridgeUS}
%\usetheme{Copenhagen}
%\usetheme{Darmstadt}
%\usetheme{Dresden}
%\usetheme{Frankfurt}
\usetheme{Goettingen}
%\usetheme{Hannover}
%\usetheme{Ilmenau}
%\usetheme{JuanLesPins}
%\usetheme{Luebeck}
%\usetheme{Madrid}
%\usetheme{Malmoe}
%\usetheme{Marburg}
%\usetheme{Montpellier}
%\usetheme{PaloAlto}
%\usetheme{Pittsburgh}
%\usetheme{Rochester}
%\usetheme{Singapore}
%\usetheme{Szeged}
%\usetheme{Warsaw}

% As well as themes, the Beamer class has a number of color themes
% for any slide theme. Uncomment each of these in turn to see how it
% changes the colors of your current slide theme.

%\usecolortheme{albatross}
%\usecolortheme{beaver}
%\usecolortheme{beetle}
%\usecolortheme{crane}
%\usecolortheme{dolphin}
%\usecolortheme{dove}
%\usecolortheme{fly}
%\usecolortheme{lily}
%\usecolortheme{orchid}
%\usecolortheme{rose}
%\usecolortheme{seagull}
\usecolortheme{seahorse}
%\usecolortheme{whale}
%\usecolortheme{wolverine}

%\setbeamertemplate{footline} % To remove the footer line in all slides uncomment this line
%\setbeamertemplate{footline}[page number] % To replace the footer line in all slides with a simple slide count uncomment this line

%\setbeamertemplate{navigation symbols}{} % To remove the navigation symbols from the bottom of all slides uncomment this line
}

\usepackage{graphicx} % Allows including images
\usepackage{booktabs} % Allows the use of \toprule, \midrule and \bottomrule in tables

\usepackage[version=latest]{pgf}
\usepackage{tikz}
\usepackage{xspace}
\usepackage{adjustbox} 

\newcommand{\tikzname}{Ti\textit{k}Z\xspace}

% For bild_des-overview image
\definecolor{grey}{HTML}{F2F2F2}
\definecolor{greyDark}{HTML}{E1E1E1}

% Define additional colors
\definecolor{brewergreen}{HTML}{a1d99b}
\definecolor{matrixblue}{HTML}{085799}
\definecolor{brewerpurple}{HTML}{756bb1}

% Transformer plots
\definecolor{transorange}{HTML}{ffe2bb} 
\definecolor{transyellow}{HTML}{f2f4c1} 
\definecolor{transblue}{HTML}{c2e8f7} 
\definecolor{transbackground}{HTML}{f3f3f4} 
\definecolor{transred}{HTML}{fce0e1} 
\definecolor{transpurple}{HTML}{dbdfef} 
\definecolor{transgreen}{HTML}{cce7cf} 

\definecolor{transredintensive}{HTML}{fc5d63} 
\definecolor{transorangeintensive}{HTML}{ebac58} 
\definecolor{transgreenintensive}{HTML}{54955b} 

\usepackage{fancybox}
% We need lots of libraries...
\usetikzlibrary{
  3d,
  arrows,
  arrows.spaced,
  arrows.meta,
  bending,
  babel,
  calc,
  fit,
  patterns,
  patterns.meta,
  plotmarks,
  shapes.geometric,
  shapes.misc,
  shapes.symbols,
  shapes.arrows,
  shapes.callouts,
  shapes.multipart,
  shapes.gates.logic.US,
  shapes.gates.logic.IEC,
  circuits.logic.US,
  circuits.logic.IEC,
  circuits.logic.CDH,
  circuits.ee.IEC,
  datavisualization,
  datavisualization.polar,
  datavisualization.formats.functions,
  er,
  automata,
  backgrounds,
  chains,
  topaths,
  trees,
  petri,
  mindmap,
  matrix,
  calendar,
  folding,
  fadings,
  shadings,
  spy,
  through,
  turtle,
  positioning,
  scopes,
  decorations.fractals,
  decorations.shapes,
  decorations.text,
  decorations.pathmorphing,
  decorations.pathreplacing,
  decorations.footprints,
  decorations.markings,
  shadows,
  lindenmayersystems,
  intersections,
  fixedpointarithmetic,
  fpu,
  svg.path,
  external,
  graphs,
  graphs.standard,
  quotes,
  math,
  angles,
  views,
  animations,
  rdf,
  perspective,
}

%----------------------------------------------------------------------------------------
%	TITLE PAGE
%----------------------------------------------------------------------------------------

\title[TikZ - Beginner's Guide]{Schöne Grafiken mit TikZ - A Beginner's Guide} % The short title appears at the bottom of every slide, the full title is only on the title page

\author{Michael Altenhuber} % Your name
\institute[FH-LUG] % Your institution as it will appear on the bottom of every slide, may be shorthand to save space
{
FH-LUG \\ % Your institution for the title page
\medskip
\textit{michael@altenhuber.net} % Your email address
}
\date{\today} % Date, can be changed to a custom date

\begin{document}

\begin{frame}
\titlepage % Print the title page as the first slide
\end{frame}

\begin{frame}
\frametitle{Overview} % Table of contents slide, comment this block out to remove it
\tableofcontents % Throughout your presentation, if you choose to use \section{} and \subsection{} commands, these will automatically be printed on this slide as an overview of your presentation
\end{frame}

%----------------------------------------------------------------------------------------
%	PRESENTATION SLIDES
%----------------------------------------------------------------------------------------

%------------------------------------------------
\section{Introduction} % Sections can be created in order to organize your presentation into discrete blocks, all sections and subsections are automatically printed in the table of contents as an overview of the talk
%------------------------------------------------

\begin{frame}
\frametitle{What is TikZ and what is it good for?}

\begin{quote}
    TikZ --- \textcolor{red}{T}ikZ \textcolor{red}{i}st \textit{\textcolor{red}{k}ein} \textcolor{red}{Z}eichenprogramm
\end{quote}

\begin{itemize}
    \item \LaTeX{} / PGF-based drawing framework
    \item program your graphics instead of using annoying UI-based tools
    \item \tikzname either integrates into your document or creates stand-alone vector graphics.
\end{itemize}

\end{frame}

%------------------------------------------------

\begin{frame}
\frametitle{Pros and Cons}
Pros
\begin{itemize}
\item Integrates seamlessly into \LaTeX documents
\item Sharp, precise vector graphics
\item Usage of variables, loops and other programming stuff
\item Easily version controllable (git)
\end{itemize}
\vspace{1em}
Cons
\begin{itemize}
    \item Steep learning curve
    \item drawing takes time
    \item May become confusing for larger graphics
\end{itemize}
\end{frame}


%------------------------------------------------
\section{When to use and examples}
%------------------------------------------------

\begin{frame}
    \frametitle{Examples}
       \begin{minipage}{0.25\textwidth}
        %\documentclass[12pt]{scrartcl}
%\usepackage[utf8]{inputenc}		% bei der Verw. von lualatex oder xelatex entfernen!
%\usepackage[ruled]{algorithm}
%\usepackage{mathtools}
%\usepackage{tikz}
%\usepackage{rotating}
%\usetikzlibrary{calc,shapes.geometric, positioning,backgrounds, arrows.meta, chains, plotmarks, fit, arrows}
%\usetikzlibrary{decorations.pathreplacing}
%\usetikzlibrary{intersections}
%
%\usepackage{textcomp}
%\usepackage{enumitem}
%
%\setlength{\abovedisplayskip}{3pt}
%\setlength{\belowdisplayskip}{3pt}
%
%\definecolor{grey}{HTML}{F2F2F2}
%\definecolor{greyDark}{HTML}{E1E1E1}
%\begin{document}
%\begin{figure}
%\centering
%\caption{}
\begin{tikzpicture}[scale=0.54, every node/.style={transform shape}]

\dimendef\prevdepth=0

\tikzset{half paths/.style 2 args={%
  decoration={show path construction,
    lineto code={
      \draw [#1] (\tikzinputsegmentfirst) -- 
         ($(\tikzinputsegmentfirst)!0.5!(\tikzinputsegmentlast)$);
      \draw [#2] ($(\tikzinputsegmentfirst)!0.5!(\tikzinputsegmentlast)$)
        -- (\tikzinputsegmentlast);
    }
  }, decorate
}}

\tikzset{
	smallarrow/.style={-{stealth[width=1pt]}}
}

\tikzset{
    myarrow/.style={-{stealth}, semithick},
    myarrowdashed/.style={semithick, dashed},
    myline/.style={semithick},
    triangle/.style = {draw, regular polygon, regular polygon sides=3, inner sep=0pt, minimum height=130},
    node rotated/.style = {rotate=180},
    border rotated/.style = {shape border rotate=180},
     pre/.style={<-,shorten <=1pt,>=stealth',semithick},
     post/.style={->,shorten >=1pt,>=stealth',semithick}
}

\tikzset{XOR/.style={draw,circle,append after command={
        [shorten >=\pgflinewidth, shorten <=\pgflinewidth, semithick,]
        (\tikzlastnode.north) edge (\tikzlastnode.south)
        (\tikzlastnode.east) edge (\tikzlastnode.west)
        }
    }
}

\tikzset{BOXPLUS/.style={draw,rectangle, inner sep=6pt,append after command={
        [shorten >=\pgflinewidth, shorten <=\pgflinewidth, semithick,]
        (\tikzlastnode.north) edge (\tikzlastnode.south)
        (\tikzlastnode.east) edge (\tikzlastnode.west)
        }
    }
} 


\pgfmathsetmacro{\D}{10}
\pgfmathsetmacro{\W}{120}
\pgfmathsetmacro{\H}{60}
\pgfmathsetmacro{\A}{50}
 
%******************** 
   % Beginning of drawing
%********************

%ANCHOR TOP
\node [draw, fill=gray, minimum width=\W pt] (kBlock) {Klartext-Block};

%EINGANGSPERMUTATION
\node [draw, fill=gray, minimum width=\W - 20 pt, below=\D pt of kBlock, align=center] (ip) {\small Eingangs- \\ \small permutation};


%BEGIN ROUND 1 BOX
\node [fill=grey, below=\D pt of ip.south, anchor=north, minimum width=\W pt, minimum height=\H pt](round1Block) {};



%ANCHOR TOP LEFT
\path let \p0 = ($ (kBlock.south west)!0.20!(kBlock.south east) $)
	in node [] (kanch1) at (\p0) {};

%ANCHOR TOP RIGHT	
\path let \p0 = ($ (kBlock.south west)!0.80!(kBlock.south east) $)
	in node [] (kanch2) at (\p0) {};
	

%ANCHOR TOP LEFT
\path let
	\p0	= (kanch1),
	\p1 = ($ (ip.north west)!0.20!(ip.north east) $)
	in node [] (ip1north) at (\x0, \y1) {};

%ANCHOR TOP RIGHT	
\path let
	\p0	= (kanch2),
	\p1 = ($ (ip.north west)!0.20!(ip.north east) $)
	in node [] (ip2north) at (\x0, \y1) {};

%ANCHOR TOP LEFT
\path let
	\p0	= (kanch1),
	\p1 = ($ (ip.south west)!0.20!(ip.south east) $)
	in node [] (ip1south) at (\x0, \y1) {};

%ANCHOR TOP RIGHT	
\path let
	\p0	= (kanch2),
	\p1 = ($ (ip.south west)!0.20!(ip.south east) $)
	in node [] (ip2south) at (\x0, \y1) {};




% F1
\path let \p0 = ($ (round1Block.south)!0.50!(round1Block.north) $)
	in node [draw] (f1) at (\p0) {\small $f$};
% XOR ROUND 1
\path let
	\p0 = (kanch1),
	\p1 = (f1)
	in node [XOR] (r1fconleft) at (\x0 ,\y1) {};
	

	
	
%ROUND KEY
\path let
	\p0 = (f1),
	\p1 = ($ (round1Block.south)!0.90!(round1Block.north) $)
	in node [] (roundK1) at (\x0 ,\y1) {\scriptsize Rundenschlüssel};

%Point before twist LEFT
\path let
	\p0 = (r1fconleft),
	\p1 = ($ (round1Block.south)!0.30!(round1Block.north) $)
	in node [] (r1conleft) at (\x0 ,\y1) {};

% BRANCH ABOVE RIGHT LINE
\path let
	\p0 = (kanch2),
	\p1 = (f1)
	in node [] (r1fconright) at (\x0 ,\y1) {};	
	
% DESCRIPTION
\path let
	\p0 = ($ (r1fconright.center)!0.50!(round1Block.east) $),
	\p1 = (f1)
	in node [rotate=90] (descr) at (\x0 ,\y1) {\small Runde 1};

%Point before twist LEFT
\path let
	\p0 = (kanch2),
	\p1 = ($ (round1Block.south)!0.30!(round1Block.north) $)
	in node [] (r1conright) at (\x0 ,\y1) {};
	
%TWISTER
\path let
	\p0 = (r1fconleft),
	\p1 = (round1Block.south)
	in node [] (r1twistleft) at (\x0 ,\y1) {};
	
\path let
	\p0 = (r1fconright),
	\p1 = (round1Block.south)
	in node [] (r1twistright) at (\x0 ,\y1) {};
	
%%% DRAW LINES ROUND 1
\draw[myarrow] (kanch1.center) -- (ip1north.center);
\draw[myarrow] (kanch2.center) -- (ip2north.center);
\draw[myarrow] (ip1south.center) -- (r1fconleft.north);
\draw[myline] (ip2south.center) -- (r1fconright.center);
\draw[myarrow] (r1fconright.center) -- (f1.east);
\draw[myarrow] (f1.west) -- (r1fconleft.east);
\draw[myline] (r1fconleft.south) -- (r1conleft.center);
\draw[myline] (r1fconright.center) -- (r1conright.center);
\draw[myline] (r1conleft.center) -- (r1twistright.center);
\draw[myline] (r1conright.center) -- (r1twistleft.center);
\draw[myarrow] (roundK1) -- (f1.north);


%%%%%%%%%%%%%%%%%%%%%%%%%%%%%%%%%%%%%%%%%%%%%
%BEGIN ROUND 2 BOX
%%%%%%%%%%%%%%%%%%%%%%%%%%%%%%%%%%%%%%%%%%%%%
\node [fill=greyDark, below=\D pt of round1Block.south, anchor=north, minimum width=\W pt, minimum height=\H pt](round2Block) {};

%ANCHOR TOP LEFT
\path let \p0 = ($ (round2Block.north west)!0.20!(round2Block.north east) $)
	in node [] (r2kanch1) at (\p0) {};

%ANCHOR TOP RIGHT	
\path let \p0 = ($ (round2Block.north west)!0.80!(round2Block.north east) $)
	in node [] (r2kanch2) at (\p0) {};


% F2
\path let \p0 = ($ (round2Block.south)!0.50!(round2Block.north) $)
	in node [draw] (f2) at (\p0) {\small $f$};
% XOR ROUND 1
\path let
	\p0 = (kanch1),
	\p1 = (f2)
	in node [XOR] (r2fconleft) at (\x0 ,\y1) {};
	
%ROUND KEY
\path let
	\p0 = (f2),
	\p1 = ($ (round2Block.south)!0.90!(round2Block.north) $)
	in node [] (roundK2) at (\x0 ,\y1) {\scriptsize Rundenschlüssel};

% DESCRIPTION
\path let
	\p0 = ($ (r1fconright.center)!0.50!(round1Block.east) $),
	\p1 = (f2)
	in node [rotate=90] (descr) at (\x0 ,\y1) {\small Runde 2};

%Point before twist LEFT
\path let
	\p0 = (r2fconleft),
	\p1 = ($ (round2Block.south)!0.30!(round2Block.north) $)
	in node [] (r2conleft) at (\x0 ,\y1) {};

% BRANCH ABOVE RIGHT LINE
\path let
	\p0 = (kanch2),
	\p1 = (f2)
	in node [] (r2fconright) at (\x0 ,\y1) {};	

%Point before twist LEFT
\path let
	\p0 = (kanch2),
	\p1 = ($ (round2Block.south)!0.30!(round2Block.north) $)
	in node [] (r2conright) at (\x0 ,\y1) {};
	
%TWISTER
\path let
	\p0 = (r2fconleft),
	\p1 = (round2Block.south)
	in node [] (r2twistleft) at (\x0 ,\y1) {};
	
\path let
	\p0 = (r1fconright),
	\p1 = (round2Block.south)
	in node [] (r2twistright) at (\x0 ,\y1) {};
	
	
%DASHED%
\draw[myline] (r1twistleft.center) -- (r2kanch1.center);
\draw[myline] (r1twistright.center) -- (r2kanch2.center);

%%% DRAW LINES ROUND 2
\draw[myarrow] (r2kanch1.center) -- (r2fconleft.north);
\draw[myline] (r2kanch2.center) -- (r2fconright.center);
\draw[myarrow] (r2fconright.center) -- (f2.east);
\draw[myarrow] (f2.west) -- (r2fconleft.east);
\draw[myline] (r2fconleft.south) -- (r2conleft.center);
\draw[myline] (r2fconright.center) -- (r2conright.center);
\draw[myline] (r2conleft.center) -- (r2twistright.center);
\draw[myline] (r2conright.center) -- (r2twistleft.center);
\draw[myarrow] (roundK2) -- (f2.north);


%%%%%%%%%%%%%%%%%%%%%%%%%%%%%%%%%%%%%%%%%%%%%
%BEGIN ROUND 3 BOX
%%%%%%%%%%%%%%%%%%%%%%%%%%%%%%%%%%%%%%%%%%%%%
\node [fill=greyDark, below=\D * 2 pt of round2Block.south, anchor=north, minimum width=\W pt, minimum height=\H pt](round3Block) {};

%ANCHOR TOP LEFT
\path let \p0 = ($ (round3Block.north west)!0.20!(round3Block.north east) $)
	in node [] (r3kanch1) at (\p0) {};

%ANCHOR TOP RIGHT	
\path let \p0 = ($ (round3Block.north west)!0.80!(round3Block.north east) $)
	in node [] (r3kanch2) at (\p0) {};


% F2
\path let \p0 = ($ (round3Block.south)!0.50!(round3Block.north) $)
	in node [draw] (f3) at (\p0) {\small $f$};
% XOR ROUND 1
\path let
	\p0 = (kanch1),
	\p1 = (f3)
	in node [XOR] (r3fconleft) at (\x0 ,\y1) {};
	
%ROUND KEY
\path let
	\p0 = (f3),
	\p1 = ($ (round3Block.south)!0.90!(round3Block.north) $)
	in node [] (roundK3) at (\x0 ,\y1) {\scriptsize Rundenschlüssel};

% DESCRIPTION
\path let
	\p0 = ($ (r1fconright.center)!0.50!(round1Block.east) $),
	\p1 = (f3)
	in node [rotate=90] (descr) at (\x0 ,\y1) {\small Runde 15};

%Point before twist LEFT
\path let
	\p0 = (r3fconleft),
	\p1 = ($ (round3Block.south)!0.30!(round3Block.north) $)
	in node [] (r3conleft) at (\x0 ,\y1) {};

% BRANCH ABOVE RIGHT LINE
\path let
	\p0 = (kanch2),
	\p1 = (f3)
	in node [] (r3fconright) at (\x0 ,\y1) {};	

\path let
	\p0 = (kanch2),
	\p1 = ($ (round3Block.south)!0.30!(round3Block.north) $)
	in node [] (r3conright) at (\x0 ,\y1) {};
	
%TWISTER
\path let
	\p0 = (r3fconleft),
	\p1 = (round3Block.south)
	in node [] (r3twistleft) at (\x0 ,\y1) {};
	
\path let
	\p0 = (r3fconright),
	\p1 = (round3Block.south)
	in node [] (r3twistright) at (\x0 ,\y1) {};
	
	
%DASHED%
%\draw[myarrowdashed] (r1twistleft) -- (r2kanch1.center);
%\draw[myarrowdashed] (r1twistright) -- (r2kanch2.center);

%%% DRAW LINES ROUND 3
\draw[myarrow] (r3kanch1.center) -- (r3fconleft.north);
\draw[myline] (r3kanch2.center) -- (r3fconright.center);
\draw[myarrow] (r3fconright.center) -- (f3.east);
\draw[myarrow] (f3.west) -- (r3fconleft.east);
\draw[myline] (r3fconleft.south) -- (r3conleft.center);
\draw[myline] (r3fconright.center) -- (r3conright.center);
\draw[myline] (r3conleft.center) -- (r3twistright.center);
\draw[myline] (r3conright.center) -- (r3twistleft.center);
\draw[myarrow] (roundK3) -- (f3.north);



%%DASHED LINES
\draw[myarrowdashed] (r2twistleft.center) -- (r3kanch1.center);
\draw[myarrowdashed] (r2twistright.center) -- (r3kanch2.center);
%\draw[myline] (boxplus1.south) -- (r3kanch1.center);
%\draw[myline] (boxplus2.south) -- (r3kanch2.center);


%%%%%%%%%%%%%%%%%%%%%%%%%%%%%%%%%%%%%%%%%%%%%
%BEGIN ROUND 4 BOX
%%%%%%%%%%%%%%%%%%%%%%%%%%%%%%%%%%%%%%%%%%%%%
\node [fill=greyDark, below=\D pt of round3Block.south, anchor=north, minimum width=\W pt, minimum height=\H pt](round4Block) {};

%ANCHOR TOP LEFT
\path let \p0 = ($ (round4Block.north west)!0.20!(round4Block.north east) $)
	in node [] (r4kanch1) at (\p0) {};

%ANCHOR TOP RIGHT	
\path let \p0 = ($ (round4Block.north west)!0.80!(round4Block.north east) $)
	in node [] (r4kanch2) at (\p0) {};


% F2
\path let \p0 = ($ (round4Block.south)!0.50!(round4Block.north) $)
	in node [draw] (f4) at (\p0) {\small $f$};
% XOR ROUND 1
\path let
	\p0 = (kanch1),
	\p1 = (f4)
	in node [XOR] (r4fconleft) at (\x0 ,\y1) {};
	
%ROUND KEY
\path let
	\p0 = (f4),
	\p1 = ($ (round4Block.south)!0.90!(round4Block.north) $)
	in node [] (roundK4) at (\x0 ,\y1) {\scriptsize Rundenschlüssel};
	

% DESCRIPTION
\path let
	\p0 = ($ (r1fconright.center)!0.50!(round1Block.east) $),
	\p1 = (f4)
	in node [rotate=90] (descr) at (\x0 ,\y1) {\small Runde 16};

%Point before twist LEFT
\path let
	\p0 = (r4fconleft),
	\p1 = ($ (round4Block.south)!0.30!(round4Block.north) $)
	in node [] (r4conleft) at (\x0 ,\y1) {};

% BRANCH ABOVE RIGHT LINE
\path let
	\p0 = (kanch2),
	\p1 = (f4)
	in node [] (r4fconright) at (\x0 ,\y1) {};	

\path let
	\p0 = (kanch2),
	\p1 = ($ (round4Block.south)!0.30!(round4Block.north) $)
	in node [] (r4conright) at (\x0 ,\y1) {};
	
%TWISTER
\path let
	\p0 = (r4fconleft),
	\p1 = (round4Block.south)
	in node [] (r4twistleft) at (\x0 ,\y1) {};
	
\path let
	\p0 = (r4fconright),
	\p1 = (round4Block.south)
	in node [] (r4twistright) at (\x0 ,\y1) {};
	
	
%DASHED%
\draw[myline] (r3twistleft.center) -- (r4kanch1.center);
\draw[myline] (r3twistright.center) -- (r4kanch2.center);

%%% DRAW LINES ROUND 2
\draw[myarrow] (r4kanch1.center) -- (r4fconleft.north);
\draw[myline] (r4kanch2.center) -- (r4fconright.center);
\draw[myarrow] (r4fconright.center) -- (f4.east);
\draw[myarrow] (f4.west) -- (r4fconleft.east);
\draw[myline] (r4fconleft.south) -- (r4conleft.center);
\draw[myline] (r4fconright.center) -- (r4conright.center);
\draw[myline] (r4conleft.center) -- (r4twistright.center);
\draw[myline] (r4conright.center) -- (r4twistleft.center);
\draw[myarrow] (roundK4) -- (f4.north);


%FINAL PERMUTATION
\node [draw, fill=gray, minimum width=\W - 20 pt, below=\D pt of round4Block, align=center] (fp) {\small Ausgangs- \\ \small permutation};

%ANCHOR BOTTOM
\node [draw, fill=gray, minimum width=\W pt, below=\D pt of fp.south, anchor=north] (cBlock) {Chiffrat-Block};

%ANCHOR TOP LEFT
\path let
	\p0	= (kanch1),
	\p1 = ($ (fp.north west)!0.20!(fp.north east) $)
	in node [] (fp1north) at (\x0, \y1) {};

%ANCHOR TOP RIGHT	
\path let
	\p0	= (kanch2),
	\p1 = ($ (fp.north west)!0.20!(fp.north east) $)
	in node [] (fp2north) at (\x0, \y1) {};

%ANCHOR TOP LEFT
\path let
	\p0	= (kanch1),
	\p1 = ($ (fp.south west)!0.20!(fp.south east) $)
	in node [] (fp1south) at (\x0, \y1) {};

%ANCHOR TOP RIGHT	
\path let
	\p0	= (kanch2),
	\p1 = ($ (fp.south west)!0.20!(fp.south east) $)
	in node [] (fp2south) at (\x0, \y1) {};



\path let \p0 = ($ (cBlock.north west)!0.20!(cBlock.north east) $)
	in node [] (r5kanch1) at (\p0) {};

%ANCHOR TOP RIGHT	
\path let \p0 = ($ (cBlock.north west)!0.80!(cBlock.north east) $)
	in node [] (r5kanch2) at (\p0) {};

\draw[myarrow] (r4twistleft.center) -- (fp1north.center);
\draw[myarrow] (r4twistright.center) -- (fp2north.center);
	
\draw[myarrow] (fp1south.center) -- (r5kanch1.center);
\draw[myarrow] (fp2south.center) -- (r5kanch2.center);


\end{tikzpicture}
%\end{figure}
%\end{document}
       \end{minipage}
       \begin{minipage}{0.7\textwidth}
         \begin{tikzpicture}[scale=0.30, every node/.style={transform shape}]

\dimendef\prevdepth=0

\tikzset{half paths/.style 2 args={%
  decoration={show path construction,
    lineto code={
      \draw [#1] (\tikzinputsegmentfirst) -- 
         ($(\tikzinputsegmentfirst)!0.5!(\tikzinputsegmentlast)$);
      \draw [#2] ($(\tikzinputsegmentfirst)!0.5!(\tikzinputsegmentlast)$)
        -- (\tikzinputsegmentlast);
    }
  }, decorate
}}
    
\tikzset{
    %myarrow/.style={->, shorten >=2pt, semithick},
	myarrow/.style={-{stealth}, semithick},
    myline/.style={semithick},
    triangle/.style = {draw, regular polygon, regular polygon sides=3, inner sep=0pt, minimum height=130},
    node rotated/.style = {rotate=180},
    border rotated/.style = {shape border rotate=180},
     pre/.style={<-,shorten <=1pt,>=stealth',semithick},
     post/.style={->,shorten >=1pt,>=stealth',semithick}
}

\tikzset{XOR/.style={draw,circle,append after command={
        [shorten >=\pgflinewidth, shorten <=\pgflinewidth, semithick,]
        (\tikzlastnode.north) edge (\tikzlastnode.south)
        (\tikzlastnode.east) edge (\tikzlastnode.west)
        }
    }
} 
    
\pgfmathsetmacro{\S}{8}
\pgfmathsetmacro{\M}{135}

 
%******************** 
   %P - D
%********************
	
	\node [draw, minimum width=700pt, minimum height=35pt, fill=white, text height=-3pt] (U) {U$_{1,...,36}$};
	\node [draw, minimum width=600pt,minimum height=40pt, above=30pt of U] (P) {P};

	
%Lines	
%	\foreach \i in {0,...,3} {
%		
%		\draw [myarrow] ([xshift=\i * \S - \M]P.south) -- ([xshift=\i * \S - \M]U.north);
%	}

 
%******************** 
   %Z-Funktionen
%********************

%\draw [black,dashed] (-15,11) rectangle (2,6) node [black,below] {U};



%NOTE OUT OF ORDER
\begin{scope}[start chain, node distance=70pt] 
\node[triangle, border rotated, on chain] (Z4) at ($ (P.west) + (65pt,-147.5pt) $) {Z$_{4}$};
\node[triangle, border rotated, on chain] (Z3) {Z$_{3}$};% at ($ (Z4.east) + (4,0) $) {Z$_{3}$};
\node[triangle, border rotated, on chain] (Z2) {Z$_{2}$};% at ($ (Z1.west) + (-3.2,0) $) {Z$_{2}$};
\node[triangle, border rotated, on chain] (Z1) {Z$_{1}$};%;at (P.west) {Z};%($ (P.east) + (-65pt,-6) $) {Z$_{1}$} ;
\end{scope}

\begin{scope}[on background layer]
\pgfmathsetmacro{\AS}{19} % arrow-spacing: spacing between lines horizontally
\pgfmathsetmacro{\AO}{8}  % arrow-offset: offset from center of Z anchor
\pgfmathsetmacro{\AP}{0}  % arrow-p: height of P description
\pgfmathsetmacro{\AU}{-14}% arrow-u: height of U description
\pgfmathsetmacro{\AM}{49}% arrow-m: height of middle descriptions (without Z function)
% Z4 %

\path
  let \p0 = (Z4.north west),
      \p1 = (U.south west),
      \p2 = ($ (\p0)!0.5!(\p1) + (0,0) $)
  in
node [draw=none] (S2) at (\x1, \y2) {\scriptsize  S$_{2/n}$};

\draw [myarrow] (S2)-|([xshift=-2 * \AS - \AO]Z4.north);
\draw [myarrow] ([xshift=-1 * \AS - \AO]Z4.north|-P.south) -- ([xshift=-1 * \AS - \AO]Z4.north)
	node[at start, above=\AP pt] {\scriptsize  P$_{1}$}
	node [midway, above=\AU pt] (P0) {};
\draw [myarrow] ([xshift=0 * \AS - \AO]Z4.north|-P.south) -- ([xshift=0 * \AS - \AO]Z4.north)
	node[at start, above=\AP pt] {\scriptsize  P$_{2}$}
	node [midway, above=\AU pt] (P1) {};
\path [half paths={dashed, myarrow}{solid, myarrow}] ([xshift=0 * \AS + \AO]Z4.north|-P.south)--([xshift=0 * \AS + \AO]Z4.north)
	node [midway, above=\AU pt] (P2) {};
\draw [myarrow] ([xshift=1 * \AS + \AO]Z4.north|-P.south) -- ([xshift=1 * \AS + \AO]Z4.north)
	node[at start, above=\AP pt] {\scriptsize  P$_{4}$}
	node [midway, above=\AU pt] (P3) {};
\draw [myarrow] ([xshift=2 * \AS + \AO]Z4.north|-P.south) -- ([xshift=2 * \AS + \AO]Z4.north)
	node[at start, above=\AP pt] {\scriptsize  P$_{5}$}
	node [midway, above=\AU pt] (P4) {};
	
% In Between A5 %
\path
  let \p0 = (Z4.east),
      \p1 = (Z3.west),
      \p2 = ($ (\p0)!.5!(\p1) + (0,-3.5) $)
  in
node [XOR] (A5) at (\p2) {};
\draw [myline, name path=firstline] (A5|-P.south) -- (A5)
	node[at start, above=\AP pt] {\scriptsize  P$_{6}$}
	node [midway, above=\AM pt] (P5) {};

% Z3 %
\path [half paths={dashed, myarrow}{solid, myarrow}] ([xshift=-2 * \AS - \AO]Z3.north|-P.south)--([xshift=-2 * \AS - \AO]Z3.north)
	node [midway, above=\AU pt] (P6) {};
\draw [myarrow] ([xshift=-1 * \AS - \AO]Z3.north|-P.south) -- ([xshift=-1 * \AS - \AO]Z3.north)
	node[at start, above=\AP pt] {\scriptsize  P$_{8}$}
	node [midway, above=\AU pt] (P7) {};
\path [half paths={dashed, myarrow}{solid, myarrow}] ([xshift=0 * \AS - \AO]Z3.north|-P.south) -- ([xshift=0 * \AS - \AO]Z3.north)
	%node[at start, above=\AP pt] {\scriptsize  P$_{9}$}
	node [midway, above=\AU pt] (P8) {};
\draw [myarrow] ([xshift=0 * \AS + \AO]Z3.north|-P.south) -- ([xshift=0 * \AS + \AO]Z3.north)
	node[at start, above=\AP pt] {\scriptsize  P$_{10}$}
	node [midway, above=\AU pt] (P9) {};
\draw [myarrow] ([xshift=1 * \AS + \AO]Z3.north|-P.south) -- ([xshift=1 * \AS + \AO]Z3.north)
	node[at start, above=\AP pt] {\scriptsize  P$_{11}$}
	node [midway, above=\AU pt] (P10) {};
\draw [myarrow] ([xshift=2 * \AS + \AO]Z3.north|-P.south) -- ([xshift=2 * \AS + \AO]Z3.north)
	node[at start, above=\AP pt] {\scriptsize  P$_{12}$}
	node [midway, above=\AU pt] (P11) {};
	
% In Between A15 %
\path
  let \p0 = (Z3.east),
      \p1 = (Z2.west),
      \p2 = ($ (\p0)!.5!(\p1) + (0,-3.5) $)
  in
node [XOR] (A12) at (\p2) {};
\draw [myline, name path=secondline] (A12|-P.south) -- (A12)
	node[at start, above=\AP pt] {\scriptsize  P$_{5}$}
	node [midway, above=\AM pt] (P12) {};	


% Z2 %
\draw [myarrow] ([xshift=-2 * \AS - \AO]Z2.north|-P.south) -- ([xshift=-2 * \AS - \AO]Z2.north)
	node[at start, above=\AP pt] {\scriptsize  P$_{14}$}
	node [midway, above=\AU pt] (P13) {};
\path [half paths={dashed, myarrow}{solid, myarrow}] ([xshift=-1 * \AS - \AO]Z2.north|-P.south) -- ([xshift=-1 * \AS - \AO]Z2.north)
	node [midway, above=\AU pt] (P14) {};
\draw [myarrow] ([xshift=0 * \AS - \AO]Z2.north|-P.south) -- ([xshift=0 * \AS - \AO]Z2.north)
	node[at start, above=\AP pt] {\scriptsize  P$_{16}$}
	node [midway, above=\AU pt] (P15) {};
\draw [myarrow] ([xshift=0 * \AS + \AO]Z2.north|-P.south) -- ([xshift=0 * \AS + \AO]Z2.north)
	node[at start, above=\AP pt] {\scriptsize  P$_{17}$}
	node [midway, above=\AU pt] (P16) {};
\draw [myarrow] ([xshift=1 * \AS + \AO]Z2.north|-P.south) -- ([xshift=1 * \AS + \AO]Z2.north)
	node[at start, above=\AP pt] {\scriptsize  P$_{18}$}
	node [midway, above=\AU pt] (P17) {};
\draw [myarrow] ([xshift=2 * \AS + \AO]Z2.north|-P.south) -- ([xshift=2 * \AS + \AO]Z2.north)
	node[at start, above=\AP pt] {\scriptsize  P$_{19}$}
	node [midway, above=\AU pt] (P18) {};
	
% In Between A19 %
\path
  let \p0 = (Z2.east),
      \p1 = (Z1.west),
      \p2 = ($ (\p0)!.5!(\p1) + (0,-3.5) $)
  in
node [XOR] (A19) at (\p2) {};

\draw [myline] (A19|-P.south) -- (A19)
	node[at start, above=\AP pt] {\scriptsize  P$_{20}$}
	node [midway, above=\AM pt] (P19) {};	

% Z1 %
\draw [myarrow] ([xshift=-2 * \AS - \AO]Z1.north|-P.south) -- ([xshift=-2 * \AS - \AO]Z1.north)
	node[at start, above=\AP pt] {\scriptsize  P$_{21}$}
	node [midway, above=\AU pt] (P20) {};
\draw [myarrow] ([xshift=-1 * \AS - \AO]Z1.north|-P.south) -- ([xshift=-1 * \AS - \AO]Z1.north)
	node[at start, above=\AP pt] {\scriptsize  P$_{22}$}
	node [midway, above=\AU pt] (P21) {};
\draw [myarrow] ([xshift=0 * \AS - \AO]Z1.north|-P.south) -- ([xshift=0 * \AS - \AO]Z1.north)
	node[at start, above=\AP pt] {\scriptsize  P$_{23}$}
	node [midway, above=\AU pt] (P22) {};
\path [half paths={dashed, myarrow}{solid, myarrow}] ([xshift=0 * \AS + \AO]Z1.north|-P.south) -- ([xshift=0 * \AS + \AO]Z1.north)
	node [midway, above=\AU pt] (P23) {};
\draw [myarrow] ([xshift=1 * \AS + \AO]Z1.north|-P.south) -- ([xshift=1 * \AS + \AO]Z1.north)
	node[at start, above=\AP pt] {\scriptsize  P$_{25}$}
	node [midway, above=\AU pt] (P24) {};
\draw [myarrow] ([xshift=2 * \AS + \AO]Z1.north|-P.south) -- ([xshift=2 * \AS + \AO]Z1.north)
	node[at start, above=\AP pt] {\scriptsize  P$_{26}$}
	node [midway, above=\AU pt] (P25) {};

	
% In Between %
%\draw [myarrow] ([xshift=40]Z1.north|-P.south) -- ([xshift=40]Z1.north)
%	node[at start, above=\AP pt] {\scriptsize  P$_{26}$}
%	node [midway, above=\AP pt] (P26) {};
	
%Calculate the space between Z2 and Z1 and add it to the right of Z1
\path
  let \p0 = (Z2.east),
      \p1 = (Z1.west),
      \p2 = (Z1.east),
      \p3 = ( $ (\p0)!.5!(\p1) $),
      \p4 = ( $ (\p3 ) -(\p0) $),
      \p5 = ( $ (\p4) + (\p2) $ ),
      \p6 = (  $ (\p5)  + (0, -3.5) $)
  in
node [XOR] (A26) at (\p6) {};

\draw [myline] (A26|-P.south) -- (A26)
	node[at start, above=\AP pt] {\scriptsize  P$_{27}$}
	node [midway, above=\AM pt] (P26) {};

\end{scope}

%U labels
\node[draw=none] at (P0) {\scriptsize  U$_{P_{1}}$};
\node[draw=none] at (P1) {\scriptsize  U$_{P_{2}}$};
\node[draw=none] at (P2) {\scriptsize  U$_{33}$};
\node[draw=none] at (P3) {\scriptsize  U$_{P_{4}}$};
\node[draw=none] at (P4) {\scriptsize  U$_{P_{5}}$};

\node[draw=none] at (P5) {\scriptsize  U$_{P_{6}}$};

\node[draw=none] at (P6) {\scriptsize  U$_{5}$};
\node[draw=none] at (P7) {\scriptsize  U$_{P_{8}}$};
\node[draw=none] at (P8) {\scriptsize  U$_{9}$};
\node[draw=none] at (P9) {\scriptsize  U$_{P_{10}}$};
\node[draw=none] at (P10) {\scriptsize  U$_{P_{11}}$};
\node[draw=none] at (P11) {\scriptsize  U$_{P_{12}}$};

\node[draw=none] at (P12) {\scriptsize  U$_{P_{13}}$};

\node[draw=none] at (P13) {\scriptsize  U$_{P_{14}}$};
\node[draw=none] at (P14) {\scriptsize  U$_{25}$};
\node[draw=none] at (P15) {\scriptsize  U$_{P_{16}}$};
\node[draw=none] at (P16) {\scriptsize  U$_{P_{17}}$};
\node[draw=none] at (P17) {\scriptsize  U$_{P_{18}}$};
\node[draw=none] at (P18) {\scriptsize  U$_{P_{19}}$};

\node[draw=none] at (P19) {\scriptsize  U$_{P_{20}}$};

\node[draw=none] at (P20) {\scriptsize  U$_{P_{21}}$};
\node[draw=none] at (P21) {\scriptsize  U$_{P_{22}}$};
\node[draw=none] at (P22) {\scriptsize  U$_{P_{23}}$};
\node[draw=none] at (P23) {\scriptsize  U$_{29}$};
\node[draw=none] at (P24) {\scriptsize  U$_{P_{25}}$};
\node[draw=none] at (P25) {\scriptsize  U$_{P_{26}}$};

\node[draw=none] at (P26) {\scriptsize  U$_{P_{27}}$};

% XOR nodes south of the Z triangles %
\path let \p0 = (Z4.south),\p1 = (A5)
	in node [XOR] (ZC4) at (\x0,\y1) {};

\path let \p0 = (Z3.south),\p1 = (A12)
	in node [XOR] (ZC3) at (\x0,\y1) {};

\path let \p0 = (Z2.south),\p1 = (A19)
	in node [XOR] (ZC2) at (\x0,\y1) {};

\path let \p0 = (Z1.south),\p1 = (A26)
	in node [XOR] (ZC1) at (\x0,\y1) {};
	
% draw lines south of Z triangles %	
\draw [myline] (ZC4)edge(Z4.south);
\draw [myline] (ZC3)edge(Z3.south);
\draw [myline] (ZC2)edge(Z2.south);
\draw [myline] (ZC1)edge(Z1.south);

% F line coordinates and line
\path let \p0 = (S2),\p1 = (ZC4)
	in node [] (FS) at (\x0,\y1) {F$_{n}$};
\path let \p0 = (U.east),\p1 = (ZC1)
	in node [] (FE) at (\x0,\y1) {};
\draw [myline] (FS)--(FE);

% S2 to T %
\path let
	\p0 = (S2),
	\p1 = (Z4.west),
	\p2 = ( $ (\p0)!0.5! (\p1) $)
in node [draw, circle, inner sep=1pt, fill=black] (SC1) at (\x2,\y0) {};
\path let
	\p0 = (SC1),
	\p1 = ($ (Z4.south)!0.5!(ZC4) $ )
	%\p2 = ( $ (\p0)!0.5! (\p1) $)
in node [] (SC2) at (\x0,\y1) {};
\path let
	\p0 = ($ (A12)!0.66!(ZC2) $ )
in node [XOR] (SC3) at (\p0) {};

\draw [myline] (SC1)-|(SC2.center);
\draw [myline] (SC2.center)-|(SC3);
%\draw [avoid intersection={secondline}{},avoid intersection offset=5.27pt] (SC2.center)-|(SC3);


% T 0-8 %

\path let \p0 = ($ (FS)!0.6!(ZC4) $) 
	in node [draw, circle, inner sep=1pt, fill=black] (T0N) at (\p0) {};
\node [below=30pt of T0N] (T0S) {T1};

\path let \p0 = ($ (ZC4)!0.33!(A5) $) 
	in node [draw, circle, inner sep=1pt, fill=black] (T1N) at (\p0) {};
\node [below=30pt of T1N] (T1S) {T2};

\path let \p0 = ($ (A5)!0.33!(ZC3) $) 
	in node [draw, circle, inner sep=1pt, fill=black] (T2N) at (\p0) {};
\node [below=30pt of T2N] (T2S) {T3};

\path let \p0 = ($ (ZC3)!0.33!(A12) $) 
	in node [draw, circle, inner sep=1pt, fill=black] (T3N) at (\p0) {};
\node [below=30pt of T3N] (T3S) {T4};

\path let \p0 = ($ (A12)!0.33!(ZC2) $) 
	in node [draw, circle, inner sep=1pt, fill=black] (T4N) at (\p0) {};
\node [below=30pt of T4N] (T4S) {T5};

\path let \p0 = ($ (ZC2)!0.33!(A19) $) 
	in node [draw, circle, inner sep=1pt, fill=black] (T5N) at (\p0) {};
\node [below=30pt of T5N] (T5S) {T6};

\path let \p0 = ($ (A19)!0.33!(ZC1) $) 
	in node [draw, circle, inner sep=1pt, fill=black] (T6N) at (\p0) {};
\node [below=30pt of T6N] (T6S) {T7};

\path let \p0 = ($ (ZC1)!0.33!(A26) $) 
	in node [draw, circle, inner sep=1pt, fill=black] (T7N) at (\p0) {};
\node [below=30pt of T7N] (T7S) {T8};

\path let \p0 = ($ (A26)!0.33!(FE) $) 
	in node [draw, circle, inner sep=1pt, fill=black] (T8N) at (\p0) {};
\node [below=30pt of T8N] (T8S) {T9};


\draw [myarrow] (T0N) -- (T0S);
\draw [myarrow] (T1N) -- (T1S);
\draw [myarrow] (T2N) -- (T2S);
\draw [myarrow] (T3N) -- (T3S);
\draw [myarrow] (T4N) -- (T4S);
\draw [myarrow] (T5N) -- (T5S);
\draw [myarrow] (T6N) -- (T6S);
\draw [myarrow] (T7N) -- (T7S);
\draw [myarrow] (T8N) -- (T8S);
  
  
\end{tikzpicture}

         \vspace{-5em}
         \begin{figure}
            \centering
         \begin{tikzpicture}[scale=0.7, every node/.style={transform shape}]
  
\tikzset{
    myarrow/.style={-{stealth}, thick},
    myarrowdashed/.style={semithick, dashed},
    myline/.style={semithick},
    triangle/.style = {draw, regular polygon, regular polygon sides=3, inner sep=0pt, minimum height=130},
    node rotated/.style = {rotate=180},
    border rotated/.style = {shape border rotate=180},
     pre/.style={<-,shorten <=1pt,>=stealth',semithick},
     post/.style={->,shorten >=1pt,>=stealth',semithick}
}

 
%************************%
%  Beginning of drawing  %
%************************%


\pgfmathsetmacro{\dist}{20}
\pgfmathsetmacro{\shift}{4}
\pgfmathsetmacro{\linearwidth}{38}
\pgfmathsetmacro{\linearheight}{16}
\pgfmathsetmacro{\attentionwidth}{128}
\pgfmathsetmacro{\attentionheight}{32}


\node[minimum height=16pt] (Q) {\(Q\)};
\node[left=30pt of Q, minimum height=16pt] (K) {\(K\)};
\node[right=30pt of Q, minimum height=16pt] (V) {\(V\)};


%%% Anchors
\node[above=\dist pt of Q] (linear-anchor-q) {$\circ$};
\node[above=\dist pt of K] (linear-anchor-k) {$\circ$};
\node[above=\dist pt of V] (linear-anchor-v) {$\circ$};


%%% Layer -2
\node[
  draw, minimum width=\linearwidth pt, minimum height=\linearheight pt, rounded corners, very thick, fill=transbackground, above right=2*\shift pt of linear-anchor-q.center, opacity=0.3, anchor=center
] (linear-q-2) {};
\node[
  draw, minimum width=\linearwidth pt, minimum height=\linearheight pt, rounded corners, very thick, fill=transbackground, above right=2*\shift pt of linear-anchor-k.center, opacity=0.3, anchor=center
] (linear-k-2) {};
\node[
  draw, minimum width=\linearwidth pt, minimum height=\linearheight pt, rounded corners, very thick, fill=transbackground, above right=2*\shift pt of linear-anchor-v.center, opacity=0.3, anchor=center
] (linear-v-2) {};

%%% Layer -1
\node[
  draw, minimum width=\linearwidth pt, minimum height=\linearheight pt, rounded corners, very thick, fill=transbackground, above right=\shift pt of linear-anchor-q.center, opacity=0.6, anchor=center
] (linear-q-1) {};
\node[
  draw, minimum width=\linearwidth pt, minimum height=\linearheight pt, rounded corners, very thick, fill=transbackground, above right=\shift pt of linear-anchor-k.center, opacity=0.6, anchor=center
] (linear-k-1) {};
\node[
  draw, minimum width=\linearwidth pt, minimum height=\linearheight pt, rounded corners, very thick, fill=transbackground, above right=\shift pt of linear-anchor-v.center, opacity=0.6, anchor=center
] (linear-v-1) {};

%%% Layer -1
\node[
  draw, minimum width=\linearwidth pt, minimum height=\linearheight pt, rounded corners, very thick, fill=transbackground, above right=0pt of linear-anchor-q.center, opacity=1.0, anchor=center
] (linear-q) at (linear-anchor-q) {Linear};
\node[
  draw, minimum width=\linearwidth pt, minimum height=\linearheight pt, rounded corners, very thick, fill=transbackground, opacity=1.0, anchor=center
] (linear-k)  at (linear-anchor-k) {Linear};
\node[
  draw, minimum width=\linearwidth pt, minimum height=\linearheight pt, rounded corners, very thick, fill=transbackground, opacity=1.0, anchor=center
] (linear-v)  at (linear-anchor-v) {Linear};


\node[above=1.3*\dist pt of linear-q] (attention-anchor) {$\circ$};


\node[
  draw, minimum width=\attentionwidth pt, minimum height=\attentionheight pt, rounded corners, very thick, fill=transpurple, above right=2*\shift pt of attention-anchor.center, opacity=0.3, anchor=center
] (attention-2) {};
\node[
  draw, minimum width=\attentionwidth pt, minimum height=\attentionheight pt, rounded corners, very thick, fill=transpurple, above right=\shift pt of attention-anchor.center, opacity=0.6, anchor=center
] (attention-1) {};
\node[
  draw, minimum width=\attentionwidth pt, minimum height=\attentionheight pt, rounded corners, very thick, fill=transpurple, above right=0 pt of attention-anchor.center, opacity=1, anchor=center, align=center
] (attention) {Scaled Dot-Product\\Attention};

\node[
  above=\dist pt of attention, draw, minimum width=1.2*\linearwidth pt, rounded corners, very thick, fill=transyellow, opacity=1, anchor=center, align=center
] (concat) {Concat};

\node[
  above=\dist pt of concat, draw, minimum width=1.2*\linearwidth pt, rounded corners, very thick, fill=transgreen, opacity=1, anchor=center, align=center
] (linear) {Linear};


%%% ARROWS %%%

%% Arrows vkq to linear
\draw[myarrow] (V.north) -> (linear-v);
\draw[myarrow] (K.north) -> (linear-k);
\draw[myarrow] (Q.north) -> (linear-q);

%% Arrows between linear and scaled-dot layer -2
\draw[thick, opacity=0.3] (linear-v-2.north) -| (linear-v-2.north |- attention.south);
\draw[thick, opacity=0.3] (linear-k-2.north) -| (linear-k-2.north |- attention.south);
\draw[thick, opacity=0.3] (linear-q-2.north) -| (linear-q-2.north |- attention.south);

%% Arrows between linear and scaled-dot layer -1
\draw[thick, opacity=0.6] (linear-v-1.north) -| (linear-v-1.north |- attention.south);
\draw[thick, opacity=0.6] (linear-k-1.north) -| (linear-k-1.north |- attention.south);
\draw[thick, opacity=0.6] (linear-q-1.north) -| (linear-q-1.north |- attention.south);

%% Arrows between linear and scaled-dot layer 0
\draw[myarrow] (linear-v.north) -| (linear-v.north |- attention.south);
\draw[myarrow] (linear-k.north) -| (linear-k.north |- attention.south);
\draw[myarrow] (linear-q.north) -| (linear-q.north |- attention.south);


%% Arrows between scaled-dot and concat
\draw[myarrow, opacity=0.3] (attention-2.north) -| (attention-2.north |- concat.south);
\draw[myarrow, opacity=0.6] (attention-1.north) -| (attention-1.north |- concat.south);
\draw[myarrow, opacity=1.0] (attention.north) -| (attention.north |- concat.south);

%% Arrow between concat and linear
\draw[myarrow] (concat.north) -- (linear);

%% Draw "Layer" indication

\path let
  \p1 = ($(attention.north)!0.7!(attention.south)$),
  \p2 = ($(attention.east)!0.03!(attention.west)$),
  in node[] at (\x2,\y1) (anchor-front) {};

\path let
  \p1 = ($(attention-2.north)!0.6!(attention-2.south)$),
  \p2 = (attention-2.east),
  in node[] at (\x2,\y1) (anchor-back) {};

\path let
  \p1 = ($(anchor-front)!0.5!(anchor-back)$),
  in node[] at (\x1+3pt,\y1+0pt) (anchor-middle) {};

\draw[thick] (anchor-front.center) to[out=320, in=220] (anchor-middle.center);
\draw[thick] (anchor-middle.center) to[out=40, in=0] (anchor-back.center);

\node[right=6pt of anchor-middle.center, inner sep=0] (encoder-count) {\(H_n\)};
\draw[] (encoder-count.west) -- (anchor-middle.center);

%******************%
%  End of drawing  %
%******************%

\end{tikzpicture}
         \end{figure}
       \end{minipage}
\end{frame}

%------------------------------------------------

\begin{frame}
    \ovalbox{%
\begin{tikzpicture}[level/.style={sibling distance=55mm/#1}, scale=0.60, every node/.style={transform shape}]
\node [circle,draw] (z){$n$}
  child {node [circle,draw] (a) {$\frac{n}{2}$}
    child {node [circle,draw] (b) {$\frac{n}{2^2}$}
      child {node {$\vdots$}
        child {node [circle,draw] (d) {$\frac{n}{2^k}$}}
        child {node [circle,draw] (e) {$\frac{n}{2^k}$}}
      } 
      child {node {$\vdots$}}
    }
    child {node [circle,draw] (g) {$\frac{n}{2^2}$}
      child {node {$\vdots$}}
      child {node {$\vdots$}}
    }
  }
  child {node [circle,draw] (j) {$\frac{n}{2}$}
    child {node [circle,draw] (k) {$\frac{n}{2^2}$}
      child {node {$\vdots$}}
      child {node {$\vdots$}}
    }
  child {node [circle,draw] (l) {$\frac{n}{2^2}$}
    child {node {$\vdots$}}
    child {node (c){$\vdots$}
      child {node [circle,draw] (o) {$\frac{n}{2^k}$}}
      child {node [circle,draw] (p) {$\frac{n}{2^k}$}
        child [grow=right] {node (q) {$=$} edge from parent[draw=none]
          child [grow=right] {node (q) {$O_{k = \lg n}(n)$} edge from parent[draw=none]
            child [grow=up] {node (r) {$\vdots$} edge from parent[draw=none]
              child [grow=up] {node (s) {$O_2(n)$} edge from parent[draw=none]
                child [grow=up] {node (t) {$O_1(n)$} edge from parent[draw=none]
                  child [grow=up] {node (u) {$O_0(n)$} edge from parent[draw=none]}
                }
              }
            }
            child [grow=down] {node (v) {$O(n \cdot \lg n)$}edge from parent[draw=none]}
          }
        }
      }
    }
  }
};
\path (a) -- (j) node [midway] {+};
\path (b) -- (g) node [midway] {+};
\path (k) -- (l) node [midway] {+};
\path (k) -- (g) node [midway] {+};
\path (d) -- (e) node [midway] {+};
\path (o) -- (p) node [midway] {+};
\path (o) -- (e) node (x) [midway] {$\cdots$}
  child [grow=down] {
    node (y) {$O\left(\displaystyle\sum_{i = 0}^k 2^i \cdot \frac{n}{2^i}\right)$}
    edge from parent[draw=none]
  };
\path (q) -- (r) node [midway] {+};
\path (s) -- (r) node [midway] {+};
\path (s) -- (t) node [midway] {+};
\path (s) -- (l) node [midway] {=};
\path (t) -- (u) node [midway] {+};
\path (z) -- (u) node [midway] {=};
\path (j) -- (t) node [midway] {=};
\path (y) -- (x) node [midway] {$\Downarrow$};
\path (v) -- (y)
  node (w) [midway] {$O\left(\displaystyle\sum_{i = 0}^k n\right) = O(k \cdot n)$};
\path (q) -- (v) node [midway] {=};
\path (e) -- (x) node [midway] {+};
\path (o) -- (x) node [midway] {+};
\path (y) -- (w) node [midway] {$=$};
\path (v) -- (w) node [midway] {$\Leftrightarrow$};
\path (r) -- (c) node [midway] {$\cdots$};
\end{tikzpicture}
}
\end{frame}

%------------------------------------------------

\begin{frame}
    \begin{figure}
        
\tikzset{partial ellipse/.style args =
  {#1:#2:#3}{insert path={+ (#1:#3) arc (#1:#2:#3)}}}
\begin{tikzpicture}[>=latex, scale=0.9, every node/.style={transform shape}]
  %  ellipses
  \draw [fill=white!90!red]    (3,-1.8) ellipse    (4cm and 1 cm);
  \draw [fill=yellow!90!green] (3,-1.8) ellipse (3cm and 0.75 cm);
  \draw [fill=white!90!green]  (3,-1.8) ellipse  (2cm and 0.5 cm);

  % -- Soleil
  \shade [ball color=gray!10!yellow] (3,-1.8) circle (1);
  \node (soleil) at (3,-1.8) {\bf Soleil};
  % partial ellipse pour tracé devant le Soleil
  \draw (3,-1.8) [partial ellipse=220:320:2cm and 0.5cm]
        (3,-1.8) [partial ellipse=220:320:3cm and 0.75cm];

  % Venus
  \shade [ball color=gray!10!orange] (1.6,-1.8) circle (.2);
  \node (venus) at (1.5,-1.45) {Venus}; 

  % ombre de Venus
  \draw[color=white!70!black,fill=white!70!black]
    (1.6,-2.3) ellipse (2mm and 0.5mm);

  % Mercure
  \shade [ball color=gray!10!orange] (5,-1.225) circle (.25);
  \node (mercure) at (5,-0.8) {Mercure}; 

  % Earth
  \shade [ball color=white!50!blue] (5.75,-2.5) circle (.33);
  \node (terre) at (6.6,-2.6) {\bf Terre};

  % Lune
  \shade [ball color=yellow] (5.25,-2.8) circle (.1);
  \node (lune) at (5.25,-3) {Lune};
     
  % Mars
  \draw (3,-1.8) [partial ellipse=45:120:9cm and 2.5cm];
  \shade [ball color=black!50!red] (5,0.66) circle (.15);
  \node (mars) at (5,1) {\bf Mars};   
  % trajet
  \draw [line width=2pt,blue,->,>=latex] (terre) to[out=0,in=0] (mars);   
\end{tikzpicture}
    \end{figure}
\end{frame}

%------------------------------------------------

\begin{frame}
    \frametitle{When and when not to use}

    \begin{itemize}
        \item Flowcharts
        \item Trees
        \item Schematic representations
        \item Annotate/draw over existing images
        \item Bonus (Whatever libraries there may be) - many scientific packages.
     \end{itemize}

     \begin{block}{When not to use...}
     \begin{itemize}
         \item Plots and charts (use PGPlot, matplotlib or any other plotting library)
         \item Quick Drafts
         \item Art, Fancy stuff?
     \end{itemize}
    \end{block}
\end{frame}


%------------------------------------------------
\section{Setup}
%------------------------------------------------

\begin{frame}[fragile]
\frametitle{Setup ()}
\begin{block}{\LaTeX environment}
    \begin{itemize}
        \item Most \LaTeX distributions include \tikzname
        \item May require install on certain distributions
        \subitem Miktek: Open Miktek Console \rightarrow Packages \rightarrow Install package tikz
        \subitem Others: Please let google help you
        \item Online-Editors like Overleaf usually support out-of-the-box
    \end{itemize}
\end{block}

\begin{block}{Preamble}
    \verbatim{\usepackage{tikz}}
    \verbatim{\usetikzlibrary{tikz.positioning}}
\end{block}

\end{frame}

%------------------------------------------------

\begin{frame}
\frametitle{Create a \tikzname image}
\begin{columns}[l] % The "c" option specifies centered vertical alignment while the "t" option is used for top vertical alignment

\column{.45\textwidth} % Left column and width
\textbf{inline}
\begin{lstlisting}
    \begin{figure}
        \begin{tikzpicture}[options]
            ...
        \end{tikzpicture}
    \end{figure}
\end{lstlisting}

\column{.45\textwidth} % Right column and width
\begin{lstlisting}
    \begin{figure}
        \include{images/separate_image_tex_file}
    \end{figure}
\end{lstlisting}
\end{columns}
\end{frame}

%------------------------------------------------
\section{Let's start}
%------------------------------------------------

\begin{frame}
    \frametitle{Let's start with nodes!}

    Go to guide/00_basics/00_node_basics and open the tex file.

    Alternatively go to Overleaf and do the same

\end{frame}

%------------------------------------------------

\begin{frame}
    \frametitle{Another basic command: draw}

    Go to guide/00_basics/01_draw_basics and open the tex file.

    Alternatively go to Overleaf and do the same

\end{frame}

%------------------------------------------------

\begin{frame}
    \frametitle{Another basic command: draw}

    Go to guide/00_basics/01_draw_basics and open the tex file.

    Alternatively go to Overleaf and do the same

\end{frame}

%------------------------------------------------

\begin{frame}
\frametitle{Theorem}
\begin{theorem}[Mass--energy equivalence]
$E = mc^2$
\end{theorem}
\end{frame}

%------------------------------------------------

\begin{frame}[fragile] % Need to use the fragile option when verbatim is used in the slide
\frametitle{Verbatim}
\begin{example}[Theorem Slide Code]
\begin{verbatim}
\begin{frame}
\frametitle{Theorem}
\begin{theorem}[Mass--energy equivalence]
$E = mc^2$
\end{theorem}
\end{frame}\end{verbatim}
\end{example}
\end{frame}

%------------------------------------------------

\begin{frame}
\frametitle{Figure}
Uncomment the code on this slide to include your own image from the same directory as the template .TeX file.
%\begin{figure}
%\includegraphics[width=0.8\linewidth]{test}
%\end{figure}
\end{frame}

%------------------------------------------------

\begin{frame}[fragile] % Need to use the fragile option when verbatim is used in the slide
\frametitle{Citation}
An example of the \verb|\cite| command to cite within the presentation:\\~

This statement requires citation \cite{p1}.
\end{frame}

%------------------------------------------------

\begin{frame}
\frametitle{References}
\footnotesize{
\begin{thebibliography}{99} % Beamer does not support BibTeX so references must be inserted manually as below
\bibitem[Smith, 2012]{p1} John Smith (2012)
\newblock Title of the publication
\newblock \emph{Journal Name} 12(3), 45 -- 678.
\end{thebibliography}
}
\end{frame}

%------------------------------------------------

\begin{frame}
\Huge{\centerline{The End}}
\end{frame}

%----------------------------------------------------------------------------------------

\end{document} 